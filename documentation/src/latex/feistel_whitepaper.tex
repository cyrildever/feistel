% FEISTEL white paper

%----------------------------------------------------------------------------------------
%	PACKAGES AND OTHER DOCUMENT CONFIGURATIONS
%----------------------------------------------------------------------------------------

\documentclass[twoside,twocolumn]{article}

\usepackage{blindtext} % Package to generate dummy text throughout this template 

\usepackage[sc]{mathpazo} % Use the Palatino font
\usepackage[T1]{fontenc} % Use 8-bit encoding that has 256 glyphs
%\linespread{1.05} % Line spacing - Palatino needs more space between lines
\usepackage{microtype} % Slightly tweak font spacing for aesthetics
\usepackage{eufrak}
\usepackage{graphicx} % For \scalebox

\usepackage[english]{babel} % Language hyphenation and typographical rules

\usepackage[hmarginratio=1:1,top=32mm,columnsep=20pt]{geometry} % Document margins
\usepackage[hang, small,labelfont=bf,up,textfont=it,up]{caption} % Custom captions under/above floats in tables or figures
\usepackage{booktabs} % Horizontal rules in tables

\usepackage{lettrine} % The lettrine is the first enlarged letter at the beginning of the text

\usepackage{enumitem} % Customized lists
\setlist[itemize]{noitemsep} % Make itemize lists more compact

\usepackage{abstract} % Allows abstract customization
\renewcommand{\abstractnamefont}{\normalfont\bfseries} % Set the "Abstract" text to bold
\renewcommand{\abstracttextfont}{\normalfont\small\itshape} % Set the abstract itself to small italic text

\usepackage{titlesec} % Allows customization of titles
\renewcommand\thesection{\Roman{section}} % Roman numerals for the sections
\renewcommand\thesubsection{\arabic{subsection}} % roman numerals only for subsections
\titleformat{\section}[block]{\Large\scshape\centering}{\thesection.}{1em}{} % Change the look of the section titles
\titleformat{\subsection}[block]{\large\scshape}{\thesubsection.}{1em}{} % Change the look of the section titles

\usepackage{fancyhdr} % Headers and footers
\pagestyle{fancy} % All pages have headers and footers
\fancyhead{} % Blank out the default header
\fancyfoot{} % Blank out the default footer
\fancyhead[C]{Feistel Cipher $\bullet$ Cyril Dever} % Custom header text
\fancyfoot[RO,LE]{\thepage} % Custom footer text
\setlength{\headheight}{14pt}

\usepackage{titling} % Customizing the title section

\usepackage{hyperref} % For hyperlinks in the PDF

\usepackage[symbol]{footmisc} % To use special character in footnote
\renewcommand{\thefootnote}{\arabic{footnote}}

\usepackage{outlines}
\usepackage[font=itshape]{quoting}

\usepackage[linesnumbered,ruled,vlined]{algorithm2e}
\SetKw{Continue}{continue}
\SetKw{KwBy}{by}

%----------------------------------------------------------------------------------------
%	FUNCTIONS
%----------------------------------------------------------------------------------------

\newcommand{\ceil}[1]{\left\lceil #1 \right\rceil}
\newcommand{\floor}[1]{\left\lfloor #1 \right\rfloor}
\newcommand{\bsfnote}{\textsuperscript{*}} % for reference to the base64 string note
\newcommand{\hexnote}{\textsuperscript{$\dagger$}} % for reference to the hex string note
\newcommand{\mod}[1]{\ \mathrm{mod}\ #1}

%----------------------------------------------------------------------------------------
%	LISTINGS
%----------------------------------------------------------------------------------------

\usepackage{amsthm}
\theoremstyle{definition}
\newtheorem{definition}{Definition}

\theoremstyle{remark}
\newtheorem*{remark}{Note}
\newtheorem*{recall}{Recall}

%----------------------------------------------------------------------------------------
%	FIGURES
%----------------------------------------------------------------------------------------

\usepackage{tikz}
\usepackage{caption}

\usetikzlibrary{shapes.geometric, arrows, calc, positioning}

\tikzstyle{startstop} = [rectangle, rounded corners, minimum width=3cm, minimum height=1cm,text centered, draw=black]
\tikzstyle{io} = [trapezium, trapezium left angle=70, trapezium right angle=110, minimum width=3cm, minimum height=1cm, text centered, text width=1.7cm, inner sep=0.4cm, draw=black]
\tikzstyle{process} = [rectangle, minimum width=3cm, minimum height=1cm, text centered, draw=black]
\tikzstyle{decision} = [diamond, minimum width=3cm, minimum height=1cm, text centered, inner sep=-0.1cm, draw=black]
\tikzstyle{arrow} = [thick,->,>=stealth]
\tikzset{XOR/.style={draw,circle,append after command={
        [shorten >=\pgflinewidth, shorten <=\pgflinewidth,]
        (\tikzlastnode.north) edge (\tikzlastnode.south)
        (\tikzlastnode.east) edge (\tikzlastnode.west)
        }
    }
}

%----------------------------------------------------------------------------------------
%	TITLE SECTION
%----------------------------------------------------------------------------------------

\usepackage[english]{datetime2}
\DTMsavedate{thedate}{2021-01-31}

\setlength{\droptitle}{-5\baselineskip} % Move the title up

\pretitle{\begin{center}\Large\bfseries}
\posttitle{\end{center}}
\title{Feistel Cipher with Hash Round Function} % Title
\author{%
    \textsc{Cyril Dever}\\ % Name
    \normalsize Edgewhere \\ % Institution
}
% \date{\today} % Leave empty to omit a date
\date{\DTMusedate{thedate}}
\renewcommand{\maketitlehookd}{%
    \begin{abstract}
        \noindent We needed an obfuscation tool to secure some data with an almost Format-Preserving Encryption schema.
    \end{abstract}
}

%----------------------------------------------------------------------------------------

\begin{document}

% Print the title
\maketitle

%----------------------------------------------------------------------------------------
%	ARTICLE CONTENTS
%----------------------------------------------------------------------------------------

\section{Introduction}

\lettrine[nindent=0em,lines=3]{W}hen ...

\section{Description}

We herein define $\mathfrak{F}$ our own implementation of a Feistel block cipher\cite{feistel:hf} we use for both encryption and decryption stages. 
It's an almost Format-Preserving Encryption scheme, "almost" because it depends on the size of the input; if the latter is of even length, then the 
output will preserve its size; otherwise, we'd pad it \textemdash~see (\ref{eq:padding}).

The formal algorithmic description provided by Wikipedia\footnote{\url{https://en.wikipedia.org/wiki/Feistel_cipher}} is as follows:
\begin{itemize}
    \item Let $N = n+1$ be the number of rounds, $K_{0},K_{1},...,K_{n}$ the keys associated with each round and $F: \omega \times \mathcal{K} 
        \mapsto \omega$ a function of the $(words \times keys)$ space to the $words$ space.
    \item For each step $i \in [0..n]$, note the encrypted word in step $i$, $m_i = L_i \mathbin\Vert R_i$ with $$
        \begin{array}{l}
            L_{i+1} = R_i \\
            R_{i+1} = L_i \oplus F(L_i,K_i) \\
        \end{array}$$
    \item $m_0 = L_0 \mathbin\Vert R_0$ is the unciphered text, and $m_{n+1} = L_{n+1} \mathbin\Vert R_{n+1}$ the ciphered word.
\end{itemize}

For the round function $F_i$, we don't actually use one different key per round but rather a single key $K$ and the \texttt{SHA-256} hash function 
applied to the byte array of the right part to which the key is bytewise-added at each round\footnote{a "masking" operation $\mu()$ being applied 
first on $K$ to make sure to use a key of the same length than the input}:
\begin{equation}
    \label{eq:roundFunction}
    \begin{array}{rl}
        F_i: \omega \times \mathcal{K}  &\to \omega \\
                                (x, K)  &\mapsto \texttt{SHA-256} \left(x + \mu(K) \right) \\
    \end{array}
\end{equation}
where $x + \mu(K)$ represents the addition of the UTF-8 charcodes of both terms\footnote{eg. \texttt{a} $\gets 61$, \texttt{b} $\gets 62 \Rightarrow 
\texttt{a} + \texttt{b} \gets 123 \mapsto \texttt{b01111011}$}.

As mentioned, in case the length of the input $in$ is odd, we add one padding character to the left beforehand because our cipher is a balanced 
implementation:
\begin{equation}
    \label{eq:padding}
    \begin{array}{l}
        in \gets\left\{
            \begin{array}{l}
                \textbf{if } |in| \mod 2 \neq 0 \\
                \quad \textbf{ then } \texttt{PAD\_CHAR}\mathbin\Vert in \\
                \textbf{else } in \\
            \end{array}
        \right. \\ \\
        \Rightarrow in := L \mathbin\Vert R \\
        \quad \textrm{ with } |L| = |R| = |in| \div 2 \\
    \end{array}
\end{equation}

The number of round $N$ is set to $10$ anywhere $\mathfrak{F}$ is used in the oblivion process.

Note that it's been proved \cite{Permutations:lr} that, for such an implementation of the Feistel block cipher, four rounds of permutations are enough 
to make it "strong", making our choice still very fast and two and half times stronger.

\begin{recall}
    One of the main advantage of using this Feistel block cipher construction is that encryption and decryption are similar:$$
        out = \mathfrak{F}(in, K) \iff in = \mathfrak{F}(out, K)
    $$
\end{recall}

Figure \ref{fig:feistel} provides a graphical representation of $\mathfrak{F}$.
\begin{figure}
    \centering\noindent
    \begin{tikzpicture} %see 'https://www.iacr.org/authors/tikz/'
        \centering \noindent
        % First two rounds
        \node[draw,thick,minimum width=1cm] (f1) at ($1*(0,-1.5cm)$)  {$F_1$};
        \node (xor1) [XOR, left of = f1, node distance = 2cm] {};
        \draw[thick,-latex] (f1) -- (xor1);
    
        \node[draw,thick,minimum width=1cm] (f2) at ($2*(0,-1.5cm)$)  {$F_2$};
        \node (xor2) [XOR, left of = f2, node distance = 2cm] {};
        \draw[thick,-latex] (f2) -- (xor2);
        
        \draw[thick,latex-latex] (f1.east) -| +(1.5cm,-0.5cm) -- ($(xor1) - (0,1cm)$) -- ($(xor1.north) - (0,1.5cm)$);
        \draw[thick] (xor1.south) -- ($(xor1)+(0,-0.5cm)$) -- ($(f1.east) + (1.5cm,-1cm)$) -- +(0,-0.5cm);
        
        \draw[thick,latex-] (f2.east) -| +(1.5cm,-0.5cm) -- ($(xor2) - (0,1cm)$);
        \draw[thick] (xor2.south) -- ($(xor2)+(0,-0.5cm)$) -- ($(f2.east) + (1.5cm,-1cm)$);
        
        \draw[thick, densely dotted] ($(f2.east) + (1.5cm,-1cm)$) -- +(0,-0.5cm);
        \draw[thick, densely dotted] ($(xor2) - (0,1cm)$) -- ($(xor2.north) - (0,1.5cm)$);
        
        % Middle text
        \node at (0,-4.5cm) {\scriptsize{for 10 rounds}};
    
        % Last two rounds
        \node[draw,thick,minimum width=1cm] (f3) at ($3*(0,-1.5cm) + (0, -.75cm)$)  {$F_{9}$};
        \node (xor3) [XOR, left of = f3, node distance = 2cm] {};
        \draw[thick,-latex] (f3) -- (xor3);
    
        \node[draw,thick,minimum width=1cm] (f4) at ($4*(0,-1.5cm) + (0, -.75cm)$)  {$F_{10}$};
        \node (xor4) [XOR, left of = f4, node distance = 2cm] {};
        \draw[thick,-latex] (f4) -- (xor4);
        
        \draw[thick,latex-latex] (f3.east) -| +(1.5cm,-0.5cm) -- ($(xor3) - (0,1cm)$) -- ($(xor3.north) - (0,1.5cm)$);
        \draw[thick] (xor3.south) -- ($(xor3)+(0,-0.5cm)$) -- ($(f3.east) + (1.5cm,-1cm)$) -- +(0,-0.5cm);
        
        \draw[thick, densely dotted] ($(f3.east) + (1.5cm,0cm)$) -- +(0cm,0.5cm);
        \draw[thick, densely dotted] (xor3.north) -- +(0cm,0.35cm);
    
        %% Inputs    
        \node (p0) [draw,thick,above of = f1, minimum width=5cm,minimum height=0.5cm,node distance=1cm] {$in$}; 
        \node (l0) [above of = xor1,node distance=2cm] {$L$};
        \node (r0) [right of = l0, node distance = 4cm] {$R$};
        \draw[thick,-latex] (l0 |- p0.south) -- (xor1.north);
        \draw[thick] ($(f1.east)+(1.5cm,0)$) -- +(0,0.75cm);
    
        \draw[thick,latex-] (l0 |- p0.north) -- (l0);
        \draw[thick,latex-] (r0 |- p0.north) -- (r0);
    
        %% Outputs
        \node (p4) [draw,thick,below of = f4, minimum width=5cm,minimum height=0.5cm,node distance=1cm] {$out$}; 
        \node (l4) [below of = xor4,node distance=2cm] {$L'$};
        \node (r4) [right of = l4, node distance = 4cm] {$R'$};
        \draw[thick,latex-latex] (f4.east) -| +(1.5cm,-0.75cm);
        \draw[thick,-latex] (xor4.south) -- ($(xor4)+(0,-0.75cm)$);
    
        \draw[thick,-latex] (l4 |- p4.south) -- (l4);
        \draw[thick,-latex] (r4 |- p4.south) -- (r4);
    \end{tikzpicture}
    \caption{Feistel block cipher $\mathfrak{F}$}
    \label{fig:feistel}
\end{figure}

\tableofcontents % Uncomment to add a table of contents

%----------------------------------------------------------------------------------------
%	REFERENCE LIST
%----------------------------------------------------------------------------------------

\begin{thebibliography}{99} % Bibliography

\bibitem[1]{feistel:hf}
Horst Feistel. \emph{Cryptography and Computer Privacy}, Scientific American, 1973.

\bibitem[2]{Permutations:lr}
Michael Luby, Charles Rackoff. \emph{How to Construct Pseudorandom Permutations from Pseudorandom Functions}, SIAM Journal on Computing, 1988.

\end{thebibliography}

%----------------------------------------------------------------------------------------

\end{document}