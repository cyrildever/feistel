% FEISTEL FPE white paper

%----------------------------------------------------------------------------------------
%	PACKAGES AND OTHER DOCUMENT CONFIGURATIONS
%----------------------------------------------------------------------------------------

\documentclass[twoside,twocolumn]{article}

\usepackage{blindtext} % Package to generate dummy text throughout this template 

\usepackage[sc]{mathpazo} % Use the Palatino font
\usepackage[T1]{fontenc} % Use 8-bit encoding that has 256 glyphs
%\linespread{1.05} % Line spacing - Palatino needs more space between lines
\usepackage{microtype} % Slightly tweak font spacing for aesthetics
\usepackage{eufrak}
\usepackage{graphicx} % For \scalebox

\usepackage[english]{babel} % Language hyphenation and typographical rules

\usepackage[hmarginratio=1:1,top=32mm,columnsep=20pt]{geometry} % Document margins
\usepackage[hang, small,labelfont=bf,up,textfont=it,up]{caption} % Custom captions under/above floats in tables or figures
\usepackage{booktabs} % Horizontal rules in tables

\usepackage{lettrine} % The lettrine is the first enlarged letter at the beginning of the text

\usepackage{enumitem} % Customized lists
\setlist[itemize]{noitemsep} % Make itemize lists more compact

\usepackage{abstract} % Allows abstract customization
\renewcommand{\abstractnamefont}{\normalfont\bfseries} % Set the "Abstract" text to bold
\renewcommand{\abstracttextfont}{\normalfont\small\itshape} % Set the abstract itself to small italic text

\usepackage{titlesec} % Allows customization of titles
\renewcommand\thesection{\Roman{section}} % Roman numerals for the sections
\renewcommand\thesubsection{\arabic{subsection}} % roman numerals only for subsections
\titleformat{\section}[block]{\Large\scshape\centering}{\thesection.}{1em}{} % Change the look of the section titles
\titleformat{\subsection}[block]{\large\scshape}{\thesubsection.}{1em}{} % Change the look of the section titles

\usepackage{fancyhdr} % Headers and footers
\pagestyle{fancy} % All pages have headers and footers
\fancyhead{} % Blank out the default header
\fancyfoot{} % Blank out the default footer
\fancyhead[C]{Feistel Cipher with Hash Round Function $\bullet$ Cyril Dever} % Custom header text
\fancyfoot[RO,LE]{\thepage} % Custom footer text
\setlength{\headheight}{14pt}

\usepackage{titling} % Customizing the title section

\usepackage{hyperref} % For hyperlinks in the PDF

\usepackage[symbol]{footmisc} % To use special character in footnote
\renewcommand{\thefootnote}{\arabic{footnote}}

\usepackage{outlines}
\usepackage[font=itshape]{quoting}

\usepackage[linesnumbered,ruled,vlined]{algorithm2e}
\SetKw{Continue}{continue}
\SetKw{KwBy}{by}

%----------------------------------------------------------------------------------------
%	FUNCTIONS
%----------------------------------------------------------------------------------------

\usepackage{amsmath}

\newcommand{\ceil}[1]{\left\lceil #1 \right\rceil}
\newcommand{\concat}{\mathbin{{+}\mspace{-8mu}{+}}}
\newcommand{\floor}[1]{\left\lfloor #1 \right\rfloor}
\newcommand{\bsfnote}{\textsuperscript{*}} % for reference to the base64 string note
\newcommand{\hexnote}{\textsuperscript{$\dagger$}} % for reference to the hex string note
\newcommand{\modulo}[1]{\ \mathrm{mod}\ #1}
\newcommand{\norm}[1]{\left\Vert#1\right\Vert}

%----------------------------------------------------------------------------------------
%	LISTINGS
%----------------------------------------------------------------------------------------

\usepackage{amsthm}
\theoremstyle{definition}
\newtheorem{definition}{Definition}

\theoremstyle{remark}
\newtheorem*{remark}{Note}
\newtheorem*{recall}{Recall}

%----------------------------------------------------------------------------------------
%	FIGURES
%----------------------------------------------------------------------------------------

\usepackage{tikz}
\usepackage{caption}

\usetikzlibrary{shapes.geometric, arrows, calc, positioning}

\tikzstyle{startstop} = [rectangle, rounded corners, minimum width=3cm, minimum height=1cm,text centered, draw=black]
\tikzstyle{io} = [trapezium, trapezium left angle=70, trapezium right angle=110, minimum width=3cm, minimum height=1cm, text centered, text width=1.7cm, inner sep=0.4cm, draw=black]
\tikzstyle{process} = [rectangle, minimum width=3cm, minimum height=1cm, text centered, draw=black]
\tikzstyle{decision} = [diamond, minimum width=3cm, minimum height=1cm, text centered, inner sep=-0.1cm, draw=black]
\tikzstyle{arrow} = [thick,->,>=stealth]
\tikzset{XOR/.style={draw,circle,append after command={
        [shorten >=\pgflinewidth, shorten <=\pgflinewidth,]
        (\tikzlastnode.north) edge (\tikzlastnode.south)
        (\tikzlastnode.east) edge (\tikzlastnode.west)
        }
    }
}

%----------------------------------------------------------------------------------------
%	TITLE SECTION
%----------------------------------------------------------------------------------------

\usepackage[english]{datetime2}
\DTMsavedate{thedate}{2021-03-27} % Elaborates on V1

\setlength{\droptitle}{-5\baselineskip} % Move the title up

\pretitle{\begin{center}\Large\bfseries}
\posttitle{\end{center}}
\title{Format-Preserving Encryption \\using Hash Functions \\in a Feistel Cipher} % Title
\author{%
    \textsc{Cyril Dever}\\ % Name
    \normalsize Edgewhere \\ % Institution
}
% \date{\today} % Leave empty to omit a date
\date{\DTMusedate{thedate}}
\renewcommand{\maketitlehookd}{%
    \begin{abstract}
        \noindent We define an obfuscation tool to secure data with a true Format-Preserving Encryption process. By implementing a Feistel block 
        cipher with a round function using any robust hashing function, it provides you with a tool to obfuscate strings in a safe, fast and secure way
        featuring the convenience of format-preserving. It might be used to create digital confidential redacted documents.
    \end{abstract}
}

%----------------------------------------------------------------------------------------

\begin{document}

% Print the title
\maketitle

%----------------------------------------------------------------------------------------
%	ARTICLE CONTENTS
%----------------------------------------------------------------------------------------

\section{Introduction}

\lettrine[nindent=0em,lines=3]{P}rovided you need a robust obfuscation tool for protecting your data more than an actual encryption cipher, meet our 
newest implementation of the well-known Feistel network algorithm.

It is secure yet very fast and comes with the very handy feature of Format-Preserving Encryption (FPE), ie. it takes a string and returns a string, 
and the length of the output is the same as the one of the output. The latter is said "readable" in the sense that it uses a possibly readable 
character set (getting rid of any weird control character). Our implementation uses a specific extract of the first 512-characters UTF-8 table.

When used within a document you wish to anonymize for instance, it preserves the general form of the document, only obfuscating the needed data.
For example, the company name in "Edgewhere is a technology service company" would output as "Ki(\#q|r5* is a technology service company".

But further development could even allow a full anonymization by changing each and every occurrence to a different output, making it impossible to link 
the pseudonyms to their underlying value. It becomes the digital version of redacting classified information\footnote{Please ask for my 
\texttt{redacted} library available in different language implementations for that purpose}.

\section{The Algorithm}

\subsection{Formal Description}

We herein define $\mathfrak{F}$ our own implementation of a Feistel block cipher\cite{feistel:hf}. 

We use an unbalanced implementation, cutting the input data into two parts, eventually preparing them before processing them through our round function 
(see section \ref{roundFunction}), to finally concatenating the end results to form the final obfuscated ciphertext.

It is a fully Format-Preserving Encryption scheme.

Let us start with what we use as the basis for our own implementation: the formal description provided by Wikipedia
\footnote{\url{https://en.wikipedia.org/wiki/Feistel_cipher}} for a Feistel block cipher is as described in Algorithm \ref{algo:feistel}.

Let $N = n+1$ be the number of rounds, $K_{0},K_{1},...,K_{n}$ the keys associated with each round and $F: \omega \times \mathcal{K} \mapsto \omega$ a 
function of the $(words \times keys)$ space to the $words$ space.
\begin{algorithm}
    \KwIn{a message $m$}
    \KwOut{the ciphertext $c$}
    let the encrypted word in step $i$ be $m_i := L_i \concat R_i$ with $m_0 := L_0 \concat R_0$ as the unciphered message; \\
    \For{$i \gets 0$ \KwTo $n$ \KwBy $1$}{
        \qquad $L_{i+1} \gets R_i$; \\
        \qquad $R_{i+1} \gets L_i \oplus F(L_i,K_i)$; \\
    }
    $m_N := L_{n+1} \concat R_{n+1}$; \\
    \Return{$m_N$}
    \caption{Standard Feistel cipher}
    \label{algo:feistel}
\end{algorithm}

\subsection{Hash Round Function}
\label{roundFunction}

Figure \ref{fig:feistel} provides a graphical representation of our cipher $\mathfrak{F}$ in its entirety.

Our implementation takes its robustness by not actually using one different key per round, but rather a well-tested hash function\footnote{
    In our latest FPE implementation, we offer to choose between four well-known and well-tested hashing algorithms: \texttt{Blake2b}, 
    \texttt{Keccak}, \texttt{SHA-256} or \texttt{SHA-3}, all in their 256-bits versions. 
} $\mathfrak{h}()$ and a single key $K$ in its round function $F$.

\begin{figure}
    \centering\noindent
    \begin{tikzpicture} % see 'https://www.iacr.org/authors/tikz/'
        \centering \noindent
        % First two rounds
        \node[draw,thick,minimum width=1cm] (f1) at ($1*(0,-1.5cm)$)  {$F_1$};
        \node (xor1) [XOR, left of = f1, node distance = 2cm] {};
        \draw[thick,-latex] (f1) -- (xor1);
    
        \node[draw,thick,minimum width=1cm] (f2) at ($2*(0,-1.5cm)$)  {$F_2$};
        \node (xor2) [XOR, left of = f2, node distance = 2cm] {};
        \draw[thick,-latex] (f2) -- (xor2);
        
        \draw[thick,latex-latex] (f1.east) -| +(1.5cm,-0.5cm) -- ($(xor1) - (0,1cm)$) -- ($(xor1.north) - (0,1.5cm)$);
        \draw[thick] (xor1.south) -- ($(xor1)+(0,-0.5cm)$) -- ($(f1.east) + (1.5cm,-1cm)$) -- +(0,-0.5cm);
        
        \draw[thick,latex-] (f2.east) -| +(1.5cm,-0.5cm) -- ($(xor2) - (0,1cm)$);
        \draw[thick] (xor2.south) -- ($(xor2)+(0,-0.5cm)$) -- ($(f2.east) + (1.5cm,-1cm)$);
        
        \draw[thick, densely dotted] ($(f2.east) + (1.5cm,-1cm)$) -- +(0,-0.5cm);
        \draw[thick, densely dotted] ($(xor2) - (0,1cm)$) -- ($(xor2.north) - (0,1.5cm)$);
        
        % Middle text
        \node at (0,-4.5cm) {\scriptsize{for $n$ rounds}};
    
        % Last two rounds
        \node[draw,thick,minimum width=1cm] (f3) at ($3*(0,-1.5cm) + (0, -.75cm)$)  {$F_{n-1}$};
        \node (xor3) [XOR, left of = f3, node distance = 2cm] {};
        \draw[thick,-latex] (f3) -- (xor3);
    
        \node[draw,thick,minimum width=1cm] (f4) at ($4*(0,-1.5cm) + (0, -.75cm)$)  {$F_{n}$};
        \node (xor4) [XOR, left of = f4, node distance = 2cm] {};
        \draw[thick,-latex] (f4) -- (xor4);
        
        \draw[thick,latex-latex] (f3.east) -| +(1.5cm,-0.5cm) -- ($(xor3) - (0,1cm)$) -- ($(xor3.north) - (0,1.5cm)$);
        \draw[thick] (xor3.south) -- ($(xor3)+(0,-0.5cm)$) -- ($(f3.east) + (1.5cm,-1cm)$) -- +(0,-0.5cm);
        
        \draw[thick, densely dotted] ($(f3.east) + (1.5cm,0cm)$) -- +(0cm,0.5cm);
        \draw[thick, densely dotted] (xor3.north) -- +(0cm,0.35cm);
    
        %% Inputs    
        \node (p0) [draw,thick,above of = f1, minimum width=5cm,minimum height=0.5cm,node distance=1cm] {$in$}; 
        \node (l0) [above of = xor1,node distance=2cm] {$L$};
        \node (r0) [right of = l0, node distance = 4cm] {$R$};
        \draw[thick,-latex] (l0 |- p0.south) -- (xor1.north);
        \draw[thick] ($(f1.east)+(1.5cm,0)$) -- +(0,0.75cm);
    
        \draw[thick,latex-] (l0 |- p0.north) -- (l0);
        \draw[thick,latex-] (r0 |- p0.north) -- (r0);
    
        %% Outputs
        \node (p4) [draw,thick,below of = f4, minimum width=5cm,minimum height=0.5cm,node distance=1cm] {$out$}; 
        \draw[thick,latex-latex] (f4.east) -| +(1.5cm,-0.75cm);
        \draw[thick,-latex] (xor4.south) -- ($(xor4)+(0,-0.75cm)$);    
    \end{tikzpicture}
    \caption{Feistel block cipher $\mathfrak{F}$}
    \label{fig:feistel}
\end{figure}

The round function $F$ thus consists in taking the right side $R$ at each round and apply to it two operations:
\begin{itemize}
    \item Shift $K$ by the number of round;
    \item Add $R$ to the masked key $K'$ (of the shifted $K$);
    \item Hash the result $R'$ through $\mathfrak{h}()$.
\end{itemize}

\subsubsection{Shifting the key}

To shift the passed key $K$ by one character at each round, we use the shifting function $S()$.

Let $substr(x, start)$ be a function that keeps the substring of the passed $x$ from $start$ to end.
\begin{equation}
    \label{eq:shifting}
    \begin{array}{rl}
        S: \mathcal{K} \times \mathbb{N} &\to \mathcal{K} \\
            (K, i) &\mapsto substr(K \ll i, 1) \concat K[0] \\
    \end{array}
\end{equation}

That way, we build a "new" key from $K$ for $\norm{K}$ rounds, adding security to our round function.

\subsubsection{Masking the new key}

To enable the XOR part of the Feistel cipher, we have to apply a "masking" operation $\mu()$ on the input key $K$ to make it of length $l = \norm{R}$:
\begin{small}
    \begin{equation}
        \label{eq:masking}
        \begin{array}{rl}
            \mu: \mathcal{K} \times \mathbb{N} &\to \mathcal{K} \\
            (K, l) &\mapsto K' := \left\{
                \begin{array}{l}
                    \textrm{if } \norm{K} \geq l, \sum_{i=1}^l K[i] \\ \\
                    \sum_{i=1}^l (K \times \ceil{\frac{\norm{K}}{l}})[i] \\
                \end{array}
            \right.
        \end{array}
    \end{equation}
\end{small}
If the key $K$ is too long, the masking function $\mu()$ eventually cuts it, ie. only keeping the $l$-th first bytes.
And if it is too short, it multiplies it the needed number of times and cut the concatenation to the desired length $l$.

\subsubsection{Adding parts}

At each round, we add the masked key $K'$ with the right part of the previous round $R$ through the function $A()$ described in Algorithm 
\ref{algo:addition}.

Let $charcode$ be the UTF-8 character code of the concerned byte.
\begin{algorithm}
    \KwIn{$R$, $K' \gets \mu(K, \norm{R})$}
    \KwOut{$R'$}
    initialize $R' \gets \emptyset$ of length $\norm{R}$; \\
    \ForEach{charcode $i \in R$ and $i \in K'$}{
        $R'[i] := R[i] + K'[i]$;
    }
    \Return{$R'$}
    \caption{Addition function $A$}
    \label{algo:addition}
\end{algorithm}

For example, the addition of \texttt{a} with \texttt{b} gives:
\texttt{a} $\gets 61$,  \texttt{b} $\gets 62 \Rightarrow \texttt{a} + \texttt{b} \gets 123 \mapsto \texttt{b01111011}$.

\subsubsection{Wrapping it all up}

We define the final round function $F$ at round $i$ as the hash of the previous addition, the result we XOR with the left part $L$ of the previous 
round to form the new basis for the next round where $L$ and the output of $F$ are switched.
\begin{small}
    \begin{equation}
        \label{eq:roundFunction}
        \begin{array}{rl}
            F: \omega \times \mathcal{K} \times \mathbb{N}  &\to \omega \\
                (R, K, i)  &\mapsto \mathfrak{h} \left[ A \Bigg(R, \mu \Big( S(K,i), \norm{R} \Big) \Bigg) \right] \\
        \end{array}
    \end{equation}
\end{small}

To be able to respect FPE, a neutral byte may be added to either the output of $F$ or the left part before applying the XOR operation.

$F$ is applied at every round. And our implementation eventually unpads the result of $\mathfrak{F}$ by adding a final $P^{-1}()$ step at the end.

\subsection{Full cipher}

The last parameter of the whole cipher $\mathfrak{F}$ is the number of rounds $N$.
Note that it has been proved \cite{Permutations:lr} that, for such an implementation of the Feistel block cipher, four rounds of permutations are 
enough to make it "strong"\footnote{but we usually use at least 10 rounds}.

We finally define the full cipher $\mathfrak{F}$ that respects Figure \ref{fig:feistel} with $F_i = F(R, K, i)$ at round $i$ as follows:
\begin{equation}
    \label{eq:fullCipher}
    \begin{array}{rl}
        \mathfrak{F}: \omega \times \mathcal{K} \times \mathbb{N} &\to \omega \\
                        (m, K, N) &\mapsto \mathfrak{F}(P(m), K, N) \\
    \end{array}
\end{equation}

\begin{recall}
    One of the main advantage of using this Feistel block cipher construction is that encryption and decryption are similar:$$
        out = \mathfrak{F}(in, K_\gamma, n) \iff in = \mathfrak{F}(out, K_\gamma, n)
    $$
\end{recall}

\subsection{Decipher}

Unlike the classic operation of a Feistel network, respecting FPE forces us to apply different operations for ciphering and deciphering. In particular, 
in case the input is of odd length, and the number of rounds used is also odd, we need to change a little the way the split between left and right 
parts is initially made when deciphering.

Whereas, the split function generally applied keeps the smallest part to the left side (when the input is of odd length), the first split must make the 
the right part the smallest before the first round. This is done to ensure the data provided to the round function when deciphering is the same as what 
was the final parts when ciphering.

\section{Implementation}

We enriched our initial two different implementations with the new fully FPE cipher: the one in 
JavaScript\footnote{\url{https://npmjs.org/package/feistel-ceipher}} and the one in Go\footnote{\url{https://github.com/cyrildever/feistel}}, and we 
also created a specific \texttt{redacted} library to push the concept to its full achievement.

On both environments, our latest tests show no significant impact with even a $255$ round FPE cipher (a few dozens of nanoseconds at most). The results 
are in fact mostly impacted by the speed of the used hash function on the machine it is run (and obviously a little slower in the browser of a 
commodity desktop computer).

% \vfill\eject % To force break column if need be
% \tableofcontents % Uncomment to add a table of contents

%----------------------------------------------------------------------------------------
%	REFERENCE LIST
%----------------------------------------------------------------------------------------

\begin{thebibliography}{99} % Bibliography

\bibitem[1]{feistel:hf}
Horst Feistel. \emph{Cryptography and Computer Privacy}, Scientific American, 1973.

\bibitem[2]{Permutations:lr}
Michael Luby, Charles Rackoff. \emph{How to Construct Pseudorandom Permutations from Pseudorandom Functions}, SIAM Journal on Computing, 1988.

\end{thebibliography}

%----------------------------------------------------------------------------------------

\end{document}